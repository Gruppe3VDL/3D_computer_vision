\documentclass{article}

\usepackage[small,compact]{titlesec}
\usepackage[backend=biber]{biblatex}
%\usepackage[spanish]{babel}
\usepackage{epsfig}
\usepackage{array}
\usepackage{xfrac}
\usepackage{amsthm}
\usepackage{amsmath}
\usepackage{amssymb}
\usepackage{todonotes}
\usepackage{centernot}
\usepackage{textcomp}
\usepackage{blindtext}
\usepackage{centernot}
\usepackage{wasysym}
\usepackage{siunitx}
\usepackage[letterpaper]{geometry}
%\usepackage{multicol}
\usepackage{color}
%\usepackage[table]{xcolor}
\usepackage{amsfonts}
\usepackage{mathtools}
\usepackage{multirow}

\usepackage[small,it]{caption}
\usepackage{titling}
\usepackage{graphicx}
%\bibliographystyle{plain}
%\bibliographystyle{babplain}
\usepackage{filecontents}
\usepackage{titlesec}
\usepackage[section]{placeins}
\usepackage[hidelinks]{hyperref}
\usepackage{fancyhdr}
\usepackage{cancel}
\usepackage{abstract}
%\usepackage{minted}

\sisetup{output-exponent-marker=\textsc{e}}

\captionsetup[table]{name=Table}


%\usepackage[makestderr=true]{pythontex}
%\restartpythontexsession{\thesection}

%\setpythontexoutputdir{./Temp}


\addbibresource{Bibliography.bib}

\pagestyle{fancy}
\usepackage[utf8]{inputenc}
\fancyhf{}
\fancyhead[c]{\textbf{\@title}}
\fancyfoot[c]{\thepage}
\def\Section {\S}

\newcommand\tstrut{\rule{0pt}{2.4ex}}
\newcommand\bstrut{\rule[-1.0ex]{0pt}{0pt}}

\DeclareMathOperator*{\argmax}{arg\,max}
\DeclareMathOperator*{\argmin}{arg\,min}

\setlength{\droptitle}{-4em}
\newcommand{\squishlist}{
 \begin{list}{$\bullet$}
  { \setlength{\itemsep}{0pt}
     \setlength{\parsep}{3pt}
     \setlength{\topsep}{3pt}
     \setlength{\partopsep}{0pt}
     \setlength{\leftmargin}{1.5em}
     \setlength{\labelwidth}{1em}
     \setlength{\labelsep}{0.5em} } }


\newcommand{\squishlisttwo}{
 \begin{list}{$\bullet$}
  { \setlength{\itemsep}{0pt}
    \setlength{\parsep}{0pt}
    \setlength{\topsep}{0pt}
    \setlength{\partopsep}{0pt}
    \setlength{\leftmargin}{2em}
    \setlength{\labelwidth}{1.5em}
    \setlength{\labelsep}{0.5em} } }

\newcommand{\squishend}{
  \end{list}  }
\footskip = 50pt
\setlength{\skip\footins}{10pt}

\newcounter{proofc}
\renewcommand\theproofc{(\arabic{proofc})}
\DeclareRobustCommand\stepproofc{\refstepcounter{proofc}\theproofc}
\newenvironment{twoproof}{\tabular{@{\stepproofc}c|l}}{\endtabular}

%\usemintedstyle{tango}
 %% The usual stuff that sits
 %% between \documentclass
 %%    and \begin{document}

%\hypersetup{
%    bookmarks= \quadtrue,         % show bookmarks bar?
%    unicode= \quadfalse,          % non-Latin characters in Acrobat’s bookmarks
%    pdftoolbar= \quadtrue,        % show Acrobat’s toolbar?
%    pdfmenubar= \quadtrue,        % show Acrobat’s menu?
%    pdffitwindow= \quadfalse,     % window fit to page when opened
%    pdfstartview= \quad{FitH},    % fits the width of the page to the window
%    pdftitle= \quad{My title},    % title
%    pdfauthor= \quad{Author},     % author
%    pdfsubject= \quad{Subject},   % subject of the document
%    pdfcreator= \quad{Creator},   % creator of the document
%    pdfproducer= \quad{Producer}, % producer of the document
%    pdfkeywords= \quad{keyword1} {key2} {key3}, % list of keywords
%    pdfnewwindow= \quadtrue,      % links in new window
%    colorlinks= \quadfalse,       % false: boxed links; true: colored links
%    linkcolor= \quadred,          % color of internal links (change box color with linkbordercolor)
%    citecolor= \quadgreen,        % color of links to bibliography
%    filecolor= \quadmagenta,      % color of file links
%    urlcolor= \quadcyan           % color of external links
%}

%\addbibresource{References.bib}


\begin{document}
 %\thispagestyle{plain}
 \def\maketitle{%\twocolumn[%
 \thispagestyle{plain}
 \vspace{-10ex}
 \hrule
 \bigskip
 \begin{center}
 {\Large{\textbf{\@title}}}
 \end{center}
 \bigskip
 \hrule

 \bigskip

 \begin{flushleft}
 \textbf{\normalsize{Muhammad Gul Zain Ali Khan}} 
 \\
 \vspace{5pt}
 \textbf{\normalsize{Hasnat}} 
 \\
 \vspace{5pt}
 \textbf{\normalsize{Danish}}
 \\
\vspace{5pt}
 \textbf{\normalsize{Zeqiu Wu}} \hfill \texttt{zeqiuwu@rhrk.uni-kl.de}
 \\
 \vspace{5pt}
 3D Computer Vision \vspace{5pt}
\hfill \today \\ 
 \end{flushleft}
 %\hspace{265.2pt}
 %\bigskip
 %\bigskip
 }
\def\title#1{\def\@title{#1}}
\title{\textit{Exercise 3}}



% \squishlist    %% \begin{itemize}
%\item First item
%%\item Second item
%%\squishend     %% \end{itemize}
 %% The rest of the paper (with no maketitle)
\maketitle

\section{General Homography Transformation Matrix}
A homography ($\tilde{H} \in \mathbb{P}^{n \times n}$) on homogeneous coordinates is defined as the composition of a general Affine transformation $\mathbf{A} \in \mathbb{R}^{(n - 1) \times (n - 1)}$ (Rotation, Scaling, Reflection, Shearing), a translation $\mathbf{t} \in \mathbb{R}^{(n - 1) \times 1}$ and a projection $\mathbf{u} \in \mathbb{R}^{1 \times (n - 1)}$. A general representation of homography in \(P^{2\times2}\) plane is given below

\begin{alignat}{2}
&\tilde{H}_{3\times3} = \begin{bmatrix}
\mathbf{A_{2x2}} & \mathbf{t_{2x1}} \\
\mathbf{u_{2x1}}^{T} & \alpha
\end{bmatrix} & \quad   \label{eq:e1}
\end{alignat}

It is to be noted that \(\alpha\) is the scalar value which scales the last value of homogeneous coordinate vector. This only is needed when the projection is from real world. In homography matrix, we are only concerned with planar homography and hence, \(\alpha\) needs to be equals to 1. Below is then degree of freedom provided for homography in $P^{2x2}$ \newline

Degree of freedom in affine transformation $A_{2\times2}= 2\times2=4$

Degree of freedom in translation $t_{2\times1}= 2\times1= 2$

Degree of freedom in projection $u_{2\times1}= 2\times1=2$

Total Degree of freedom = $DOF(A_{2\times2})+ DOF(t_{2\times1})+ DOF(u_{2\times1})= 4+2+2= 8$ \newline

A point in n dimension is represented by n+1 dimension vector $[x_{1},x_{2},.., x_{n+1}]^{T}$. A homography of this dimensional vector would have dimensions $H_{(n+1)\times (n+1)}$ and the formation of this vector is given below:

\begin{alignat}{2}
&\tilde{H}_{(n+1)\times(n+1)} = \begin{bmatrix}
\mathbf{A_{n\times n}} & \mathbf{t_{n\times 1}} \\
\mathbf{u_{n\times 1}}^{T} & \alpha
\end{bmatrix} & \quad   \label{eq:e1}
\end{alignat}
\newline
Total degree of freedom = $DOF(A_{n\times n})+DOF(t_{n\times1})+DOF(u_{n\times1})$

$= (n \times n) + (n \times 1) + (n \times 1)= n^{2}+2n$

Simplifying the notation= $n^{2}+2n= (n+1)^{2}-1$


General Notation for Degree of freedom = $(n+1)^{2}-1$ \newline

Hence, homography in $P^{n \times n}$ has $(n+1)^{2} -1$ degree of freedom and is defined below:


\begin{alignat}{2}
&\tilde{H}_{(n+1)\times(n+1)} = \begin{bmatrix}
\mathbf{A_{nxn}} & \mathbf{t_{nx1}} \\
\mathbf{u_{nx1}}^{T} & \alpha
\end{bmatrix} & \quad   \label{eq:e1}
\end{alignat}


\section{Proof that an Homogeneous transform preserves lines}
Given a point $x$ on projective space $\mathbb{P}^2$, a line is defined using a coefficient vector $L \in \mathbb{R}^{3 \times 1}$, as shown on \eqref{eq:e2}. It is possible to proof that an homography transform, described through a projective matrix $H$ preserves lines by showing that the line coeffcients are transformed by the inverse homography matrix $B$. This implies that the resulting point $x_{h}$ after applying the homography belongs to the line defined by the new set of coefficients, therefore, the Homography preserves lines.

\begin{alignat}{3}
L^{T}x &= 0 \qquad &L, x &\in \mathbb{P}^{2 \times 1} \label{eq:e2} \\
\nonumber
\underbrace{L^{T}H^{-1}}_{B}Hx &= 0 \qquad &H &\in \mathbb{P}^{2 \times 2} \\
\nonumber
B\underbrace{Hx}_{x_h} &= 0 \qquad &B &\in \mathbb{P}^{1 \times 2} \\
\Rightarrow~ \Aboxed{Bx_h &= 0} \qquad &x_h &\in \mathbb{P}^{2 \times 1} \qed \label{eq:e3}
\end{alignat}

\section{Why does the rotation matrix computed from H2 need correction?}
Since, Homography H2 was computed after manual rotation there was a minute error in the rotation and $R_{rel}$ was not orthogonal. SVD was applied to $R_{rel}$ in this case to resolve the error and compute the correct $R_{rel}$

\section{Geometrical meaning of translation vector values}
By applying the extrinsic parameter matrix $[R | t]$ to the camera pose coordinate system origin $(0, 0, 0, 1)^T$, is possible to obtain the corrected position of the camera with respect to the world coordinate system. \textit{i.e.,} The translation vector $t$. On this case, the displacement vector establishes the real position of the camera with respect to the world and the movement should the camera have to take in order to be perpendicular to the image plane.

It is possible to infer from the negative values on the translation vector that the coordinate system is defined around the chessboard center. This implies that the chessboard corners are defined in terms of offsets from the center. \textit{i.e.,} Left coordinates are negative and Right coordinates are positive. This can happen if the homography computation was done taking into account offset coordinates from the board center instead of top-left centered coordinates.

%\begin{figure}[!htbp]
%\centering
%\includegraphics[scale=0.3]{./Assets/1.png}
%\caption{Traza de Wireshark que presenta la emisión y recepción de paquetes ICMP enviados a un conjunto de %clientes presentes en la misma red.}
%\end{figure}


%\section{Diseño de filtros ideales}

%\begin{alignat}{2}
%h &= \begin{bmatrix}
%1 & 1 & 1 \\
%1 & 1 & 1 \\
%1 & 1 & 1 
%\end{bmatrix} \label{eq:e6}
%\end{alignat}


% \begin{equation}
% \begin{aligned}
% f ~:~ &\mathbb{R} &\longrightarrow ~ &\mathbb{R} \label{eq:e6} \\
%     &t &\longmapsto ~ &f(t)
% \end{aligned}
% \end{equation}
% \begin{equation}
% \begin{aligned}
% x ~:~ &\mathbb{Z} &\longrightarrow ~ &\mathbb{R} \label{eq:e7} \\
%     &n &\longmapsto ~ &x[n]
% \end{aligned}
% \end{equation}



% \begin{figure*}[!htbp]
% \centering
% \epsfig{file=./Assets/Discrete.pdf,width=1.0\linewidth,clip=}
% \caption{Ejemplos de señales discretas}
% \label{Fig:F3}
% \end{figure*}




%\bibliography{biblios} \nocite{*}
\newpage
\nocite{*}
\printbibliography



%\newcounter{proofc}
%\renewcommand\theproofc{(\arabic{proofc})}
%\DeclareRobustCommand\stepproofc{\refstepcounter{proofc}\theproofc}
%\newenvironment{twoproof}{\tabular{@{\stepproofc}c|l}}{\endtabular}


\end{document}