\documentclass{article}

\usepackage[small,compact]{titlesec}
\usepackage[backend=biber]{biblatex}
%\usepackage[spanish]{babel}
\usepackage{epsfig}
\usepackage{array}
\usepackage{xfrac}
\usepackage{amsthm}
\usepackage{amsmath}
\usepackage{amssymb}
\usepackage{todonotes}
\usepackage{centernot}
\usepackage{textcomp}
\usepackage{blindtext}
\usepackage{centernot}
\usepackage{wasysym}
\usepackage{siunitx}
\usepackage[letterpaper]{geometry}
%\usepackage{multicol}
\usepackage{color}
%\usepackage[table]{xcolor}
\usepackage{amsfonts}
\usepackage{mathtools}
\usepackage{multirow}

\usepackage[small,it]{caption}
\usepackage{titling}
\usepackage{graphicx}
%\bibliographystyle{plain}
%\bibliographystyle{babplain}
\usepackage{filecontents}
\usepackage{titlesec}
\usepackage[section]{placeins}
\usepackage[hidelinks]{hyperref}
\usepackage{fancyhdr}
\usepackage{cancel}
\usepackage{abstract}
%\usepackage{minted}

\sisetup{output-exponent-marker=\textsc{e}}

\captionsetup[table]{name=Table}


%\usepackage[makestderr=true]{pythontex}
%\restartpythontexsession{\thesection}

%\setpythontexoutputdir{./Temp}


\addbibresource{Bibliography.bib}

\pagestyle{fancy}
\usepackage[utf8]{inputenc}
\fancyhf{}
\fancyhead[c]{\textbf{\@title}}
\fancyfoot[c]{\thepage}
\def\Section {\S}

\newcommand\tstrut{\rule{0pt}{2.4ex}}
\newcommand\bstrut{\rule[-1.0ex]{0pt}{0pt}}

\DeclareMathOperator*{\argmax}{arg\,max}
\DeclareMathOperator*{\argmin}{arg\,min}

\setlength{\droptitle}{-4em}
\newcommand{\squishlist}{
 \begin{list}{$\bullet$}
  { \setlength{\itemsep}{0pt}
     \setlength{\parsep}{3pt}
     \setlength{\topsep}{3pt}
     \setlength{\partopsep}{0pt}
     \setlength{\leftmargin}{1.5em}
     \setlength{\labelwidth}{1em}
     \setlength{\labelsep}{0.5em} } }


\newcommand{\squishlisttwo}{
 \begin{list}{$\bullet$}
  { \setlength{\itemsep}{0pt}
    \setlength{\parsep}{0pt}
    \setlength{\topsep}{0pt}
    \setlength{\partopsep}{0pt}
    \setlength{\leftmargin}{2em}
    \setlength{\labelwidth}{1.5em}
    \setlength{\labelsep}{0.5em} } }

\newcommand{\squishend}{
  \end{list}  }
\footskip = 50pt
\setlength{\skip\footins}{10pt}

\newcounter{proofc}
\renewcommand\theproofc{(\arabic{proofc})}
\DeclareRobustCommand\stepproofc{\refstepcounter{proofc}\theproofc}
\newenvironment{twoproof}{\tabular{@{\stepproofc}c|l}}{\endtabular}

%\usemintedstyle{tango}
 %% The usual stuff that sits
 %% between \documentclass
 %%    and \begin{document}

%\hypersetup{
%    bookmarks= \quadtrue,         % show bookmarks bar?
%    unicode= \quadfalse,          % non-Latin characters in Acrobat’s bookmarks
%    pdftoolbar= \quadtrue,        % show Acrobat’s toolbar?
%    pdfmenubar= \quadtrue,        % show Acrobat’s menu?
%    pdffitwindow= \quadfalse,     % window fit to page when opened
%    pdfstartview= \quad{FitH},    % fits the width of the page to the window
%    pdftitle= \quad{My title},    % title
%    pdfauthor= \quad{Author},     % author
%    pdfsubject= \quad{Subject},   % subject of the document
%    pdfcreator= \quad{Creator},   % creator of the document
%    pdfproducer= \quad{Producer}, % producer of the document
%    pdfkeywords= \quad{keyword1} {key2} {key3}, % list of keywords
%    pdfnewwindow= \quadtrue,      % links in new window
%    colorlinks= \quadfalse,       % false: boxed links; true: colored links
%    linkcolor= \quadred,          % color of internal links (change box color with linkbordercolor)
%    citecolor= \quadgreen,        % color of links to bibliography
%    filecolor= \quadmagenta,      % color of file links
%    urlcolor= \quadcyan           % color of external links
%}

%\addbibresource{References.bib}


\begin{document}
 %\thispagestyle{plain}
 \def\maketitle{%\twocolumn[%
 \thispagestyle{plain}
 \vspace{-10ex}
 \hrule
 \bigskip
 \begin{center}
 {\Large{\textbf{\@title}}}
 \end{center}
 \bigskip
 \hrule

 \bigskip

 \begin{flushleft}
 \textbf{\normalsize{Muhammad Gul Zain Ali Khan}} 
 \\
 \vspace{5pt}
 \textbf{\normalsize{Hasnat}} 
 \\
 \vspace{5pt}
 \textbf{\normalsize{Danish}}
 \\
\vspace{5pt}
 \textbf{\normalsize{Zeqiu Wu}} \hfill \texttt{zeqiuwu@rhrk.uni-kl.de}
 \\
 \vspace{5pt}
 3D Computer Vision \vspace{5pt}
\hfill \today \\ 
 \end{flushleft}
 %\hspace{265.2pt}
 %\bigskip
 %\bigskip
 }
\def\title#1{\def\@title{#1}}
\title{\textit{Exercise 3}}



% \squishlist    %% \begin{itemize}
%\item First item
%%\item Second item
%%\squishend     %% \end{itemize}
 %% The rest of the paper (with no maketitle)
\maketitle

\section{General Homography Transformation Matrix}
An homography ($\tilde{H} \in \mathbb{P}^{n \times n}$) on homogeneous coordinates is defined as the composition of a general Affine transformation $\mathbf{A} \in \mathbb{R}^{(n - 1) \times (n - 1)}$ (Rotation, Scaling, Reflection, Shearing), a translation $\mathbf{t} \in \mathbb{R}^{(n - 1) \times 1}$ and a projection $\mathbf{u} \in \mathbb{R}^{1 \times (n - 1)}$. Finally, the matrix is defined on terms of a scalar scaling factor $\alpha \in \mathbb{R}$ that accounts for the homogeneous coordinates transform. Written as a n-dimensional matrix \eqref{eq:e1}, the homography has $n^2 - 1$ degrees of freedom (DoF), dissected as it follows: the affine transformation accounts for $(n - 1)^2$ DoF, the translation transform has $n - 1$ DoF and the projection presents $n -1$ DoF. 

It is necessary to establish that the scalar term $\alpha$ does not increase the total number of DoF of the transformation, as it describes an scaling transform that only takes effect when a coordinate on projective space is projected back into Cartesian coordinates.

\begin{alignat}{2}
&\tilde{H} = \begin{bmatrix}
\mathbf{A} & \mathbf{t} \\
\mathbf{u}^{T} & \alpha
\end{bmatrix} & \quad   \label{eq:e1}
\end{alignat}

\section{Proof that an Homogeneous transform preserves lines}
Given a point $x$ on projective space $\mathbb{P}^2$, a line is defined using a coefficient vector $L \in \mathbb{R}^{3 \times 1}$, as shown on \eqref{eq:e2}. It is possible to proof that an homography transform, described through a projective matrix $H$ preserves lines by showing that the line coeffcients are transformed by the inverse homography matrix $B$. This implies that the resulting point $x_{h}$ after applying the homography belongs to the line defined by the new set of coefficients, therefore, the Homography preserves lines.

\begin{alignat}{3}
L^{T}x &= 0 \qquad &L, x &\in \mathbb{P}^{2 \times 1} \label{eq:e2} \\
\nonumber
\underbrace{L^{T}H^{-1}}_{B}Hx &= 0 \qquad &H &\in \mathbb{P}^{2 \times 2} \\
\nonumber
B\underbrace{Hx}_{x_h} &= 0 \qquad &B &\in \mathbb{P}^{1 \times 2} \\
\Rightarrow~ \Aboxed{Bx_h &= 0} \qquad &x_h &\in \mathbb{P}^{2 \times 1} \qed \label{eq:e3}
\end{alignat}


\section{Geometrical meaning of translation vector values}
By applying the extrinsic parameter matrix $[R | t]$ to the camera pose coordinate system origin $(0, 0, 0, 1)^T$, is possible to obtain the corrected position of the camera with respect to the world coordinate system. \textit{i.e.,} The translation vector $t$. On this case, the displacement vector establishes the real position of the camera with respect to the world and the movement should the camera have to take in order to be perpendicular to the image plane.

It is possible to infer from the negative values on the translation vector that the coordinate system is defined around the chessboard center. This implies that the chessboard corners are defined in terms of offsets from the center. \textit{i.e.,} Left coordinates are negative and Right coordinates are positive. This can happen if the homography computation was done taking into account offset coordinates from the board center instead of top-left centered coordinates.

%\begin{figure}[!htbp]
%\centering
%\includegraphics[scale=0.3]{./Assets/1.png}
%\caption{Traza de Wireshark que presenta la emisión y recepción de paquetes ICMP enviados a un conjunto de %clientes presentes en la misma red.}
%\end{figure}


%\section{Diseño de filtros ideales}

%\begin{alignat}{2}
%h &= \begin{bmatrix}
%1 & 1 & 1 \\
%1 & 1 & 1 \\
%1 & 1 & 1 
%\end{bmatrix} \label{eq:e6}
%\end{alignat}


% \begin{equation}
% \begin{aligned}
% f ~:~ &\mathbb{R} &\longrightarrow ~ &\mathbb{R} \label{eq:e6} \\
%     &t &\longmapsto ~ &f(t)
% \end{aligned}
% \end{equation}
% \begin{equation}
% \begin{aligned}
% x ~:~ &\mathbb{Z} &\longrightarrow ~ &\mathbb{R} \label{eq:e7} \\
%     &n &\longmapsto ~ &x[n]
% \end{aligned}
% \end{equation}



% \begin{figure*}[!htbp]
% \centering
% \epsfig{file=./Assets/Discrete.pdf,width=1.0\linewidth,clip=}
% \caption{Ejemplos de señales discretas}
% \label{Fig:F3}
% \end{figure*}




%\bibliography{biblios} \nocite{*}
\newpage
\nocite{*}
\printbibliography



%\newcounter{proofc}
%\renewcommand\theproofc{(\arabic{proofc})}
%\DeclareRobustCommand\stepproofc{\refstepcounter{proofc}\theproofc}
%\newenvironment{twoproof}{\tabular{@{\stepproofc}c|l}}{\endtabular}


\end{document}